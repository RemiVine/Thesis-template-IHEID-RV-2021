%%%%%%%%%%%%%%%%%%%%%%%%%%%%%%%%%%%%%%%%%%%%%%%%%%%%%%%%%
%%%%%%% put packages required all over the thesis %%%%%%% --> include here the ones you need
%%%%%%%%%%%%%%%%%%%%%%%%%%%%%%%%%%%%%%%%%%%%%%%%%%%%%%%%%
\usepackage[utf8]{inputenc}

%%%% EDITING OF PAGE FRAME %%%%
%\usepackage{showframe}         %This line can be used to clearly show the new margins

%\usepackage{a4wide}            % Unblock this package if you prefer the text to be wider on the page (more MS-like)
\usepackage{geometry}           % useful for margins setting
\newgeometry{vmargin={35mm}, hmargin={30mm,30mm}}   % set the margins (in a book format, it is useful not to shrink too much the margins)
\usepackage{changepage}         % useful to locally change margin (used in specimen)
\usepackage{fancyhdr}           % useful for headers editing

%%%% LINE, PARAGRAPH, IDENTATION %%%%
\setlength{\parindent}{2em}             % indentation for § start
\setlength{\parskip}{0.5em}             % paragraph spacing
\renewcommand{\baselinestretch}{1}      % line spacing

%%%% LANGUAGE %%%%
\usepackage[english]{babel}
\usepackage{dirtytalk}          % useful for quotation marks put in a casual way
\usepackage{textgreek}      
\usepackage{lipsum}

%%%% OTHER SECONDARY EDITING 
\usepackage{copyrightbox}       % useful for the page 3 of the specimen
\usepackage{textcomp}           % useful for the gensymb package
\usepackage{gensymb}            % required to put the "degree" symbol "°"C 
\usepackage{lineno}             % useful in case the line number is required (for some reason)      \linenumbers
\usepackage{tikz}               % useful to include flexible pictures/figures, etc.
\usepackage{epigraph}           % useful for quotation before chapter, etc.
\renewcommand{\textflush}{flushepinormal} % justify epigraph
\usepackage{pdflscape}          % useful to put landscape (e.g. for large tables)
\usepackage{lscape}             % useful to put landscape (e.g. for large tables)
\usepackage{wrapfig}            % useful some text wrapping around figures
\usepackage{pdfpages}           % useful to include the imprimatur
\usepackage{rotating}           % useful to rotate some content
%\usepackage{fontspec}          % useful for some other font, e.g. Arial font, but need another LaTeX

%%%% MORE CONTENT %%%%
\usepackage{longtable}          % useful for long table, crossing pages
\usepackage[flushleft]{threeparttable}
\usepackage{multirow}           % useful for table editing
\usepackage{subcaption}         % useful to edit table, figure, etc. captions
\usepackage{booktabs,caption}   % useful for caption alignment
\usepackage{graphicx}           % useful for inclusion of image and other content
\usepackage{float}              % useful for positioning of content (e.g. of tables)
\usepackage{etoolbox}           % useful if you use macro
\usepackage{epstopdf}           % useful in case you need EPS to be encapsulated into PDF

%%%% MATH SYMBOLS %%%%
\usepackage{amssymb}            % useful mathematical symbols
\usepackage{amsmath}            % useful mathematical symbols
\usepackage{mathrsfs}           % useful for mathematical fonts
\newcommand{\R}{\mathbb{R}}     % useful to give elegant letters
\usepackage{chngcntr}           % useful in case a different counting of propositions, etc. is required

%%%% ACRONYMS %%%% 
\usepackage[xindy,nonumberlist]{glossaries}  % useful for acronyms (no glossary)

%%%% Hyperreferencing %%%% 
\usepackage{hyperref}           % useful for referencing outside the present document
\hypersetup{
    colorlinks=true,
    linkcolor=black,
    filecolor=magenta,      
    urlcolor=cyan,
    citecolor=black     %blue,
}

%%%% APPENDICES %%%%
%\usepackage{appendix}
\usepackage[toc,page]{appendix}     % useful to have table of contents for appendix 
\AtBeginEnvironment{subappendices}{ % needed to have correct appendices in each article 
\chapter*{Appendix}
\addcontentsline{toc}{chapter}{Appendices}
\counterwithin{figure}{section}
\counterwithin{table}{section}
}

%%%% BIBLIOGRAPHY %%%%
\usepackage[sectionbib]{natbib}     % required to have one biblio per chapter
\usepackage{chapterbib}             % required to have one biblio per chapter
\newcommand{\stylechoice}{chicago}   % choose the style of bibliographies (list: https://gking.harvard.edu/files/natnotes2.pdf  --> apalike, chicago, etc.) 


%%%%%%%%%%%%%%%%%%%%%
%%% DO NOT CHANGE %%%
%%%%%%%%%%%%%%%%%%%%%

%%%% New command for thesis specimen %%%%
\renewcommand\maketitle
    {\vspace{5cm} 
    \begin{center}
        \begingroup \fontsize{14}{14}\sffamily\textbf{\MyTitle}  \endgroup      \\[0\baselineskip] 
        \vspace{3cm} 
        \begingroup \fontsize{13}{14}\sffamily\textbf{THESIS}                   \\[0\baselineskip]
        \vspace{0cm} 
        submitted at the Graduate Institute                                     \\[0\baselineskip] 
        \vspace{0cm} 
        in fulfillment of the requirements of the                               \\[0\baselineskip] 
        \vspace{0cm} 
        PhD degree in \ThesisMajor     \endgroup                                \\[0\baselineskip]  
        \vspace{1cm}         
        \begingroup \fontsize{13}{14}\sffamily{by                               \\[0\baselineskip] 
        \vspace{1cm}         
        \textbf{	\MyName 	\MySurname}                                     \\[0\baselineskip]  
        }                                                       \endgroup
    \end{center}
    } 
%%%%%%%%%%%%%%%%%%%%%
%%%%%%%%%%%%%%%%%%%%%